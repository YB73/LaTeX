\documentclass{article}
\usepackage[utf8]{inputenc}
\usepackage{array}
\usepackage{multirow}
\usepackage{tabu}

\title{Exercise  - Creating multi-row and multi-column tables}
\author{H.S.Rai, J. Singh and S.S.Sehra}
\date{12th March, 2016}
\begin{document}
	\maketitle	
	\section*{Tasks to be performed}
	\begin{enumerate}	
		\item Create multirow and multicolumn tables.
	\end{enumerate}
	\section*{Demonstration}
	%package multirow is used in the preamble to create multirow and multicolumn tables.
	% Multicolumn Table 
	Table \ref{table:2} is the example of multicolumn table
	\begin{table}[h]
		\centering

	\begin{tabular}{ |p{3cm}||p{3cm}|p{3cm}|p{3cm}|  }
		\hline
		\multicolumn{4}{|c|}{Country List} \\
		\hline
		Country Name     or Area Name& ISO ALPHA 2 Code &ISO ALPHA 3 Code&ISO numeric Code\\
		\hline
		Afghanistan   & AF    &AFG&   004\\
		Aland Islands&   AX  & ALA   &248\\
		Albania &AL & ALB&  008\\
		Algeria    &DZ & DZA&  012\\
		American Samoa&   AS  & ASM&016\\
		Andorra& AD  & AND   &020\\
		Angola& AO  & AGO&024\\
		\hline
		\end{tabular}
		\caption{Multicolumn Table}
		\label{table:2}
			\end{table}
			
	% Multirow Table 
	whereas	table \ref{table:3} is the example of multirow table
			\begin{table}[h]
				\centering
			\begin{tabular}{ |c|c|c|c| } 
		\hline
		col1 & col2 & col3 \\
		\hline
		\multirow{3}{4em}{Multiple row} & cell2 & cell3 \\ 
		& cell5 & cell6 \\ 
		& cell8 & cell9 \\ 
		\hline
	\end{tabular}
\caption{Multirow Table}
\label{table:3}
\end{table}

\end{document}
